\documentclass[pdf]{beamer}

\mode<presentation>{
  \usetheme{Singapore}          % others: default Singapore Warsaw
  \usecolortheme{dolphin}       % beetle beaver orchid whale dolphin
  \setbeamertemplate{sections/subsections in toc}[ball unnumbered]  % others: circle ball square
	\usepackage{graphicx}
  \usepackage{listings}
	\graphicspath{ {figures/} }
}

% preamble
\title{How do Jinja2 filters work?}
\subtitle{a short introduction to ansible filters}
\author{Frank Jung}
\institute{frankhjung@linux.com}
\date{ \today }
\logo{ \includegraphics[height=1.5cm]{logos.png} }

\lstset{
  basicstyle=\tiny\color{blue},
  frame=none,
  tabsize=2,
  showspaces=false,
  showtabs=false
}

\begin{document}

\begin{frame}
  \titlepage{}
\end{frame}

% \begin{frame}{Contents}
%   \tableofcontents{}
% \end{frame}

\begin{frame}
  \frametitle{Topics}
  Topics being covered are \ldots
  \pause{}
  \begin{itemize}
    \item{what are Ansible filters?}
      \pause{}
    \item{when and where to use filters}
      \pause{}
    \item{when documentation is not enough}
      \pause{}
    \item{should I roll my own?}
  \end{itemize}
\end{frame}

\section{What are Ansible Filters?}

\begin{frame}
  \frametitle{What are Ansible Filters?}
  Filters are from a fast, flexible, Python template engine called \ldots
  \pause{}
  \begin{center}
    \textbf{Jinja2}
    \begin{figure}
      \includegraphics[width=0.5\textwidth]{jinja-logo.png}
    \end{figure}
    \href{http://jinja.pocoo.org}{jinja.pocoo.org}
  \end{center}
\end{frame}

\begin{frame}
  \frametitle{What are Ansible Filters?}
  Filters are used to transform data \ldots
  \pause{}
  \begin{enumerate}
    \item{they can be used inside a template expression}
      \pause{}
    \item{they can be used to manipulate local data}
  \end{enumerate}
\end{frame}

\begin{frame}
  \frametitle{What are Ansible Filters?}
  Important features \ldots
  \pause{}
  \begin{enumerate}
    \item{sand boxed}
      \pause{}
      \begin{itemize}
        \item{so can be used to evaluate untrusted code}
      \end{itemize}
      \pause{}
    \item{they are executed on the Ansible controller}
      \pause{}
      \begin{itemize}
        \item{\textcolor{red}{\textbf{not}} on the task's target host}
      \end{itemize}
      \pause{}
    \item{Ansible ships with its own filters}
      \pause{}
      \begin{itemize}
        \item{or use standard filters from Jinja2}
          \pause{}
        \item{or write your own!}
      \end{itemize}
  \end{enumerate}
\end{frame}

\section{Typical Use Cases}

\begin{frame}[fragile]
  \frametitle{Typical Filter Use Cases}
  \setbeamercolor{normal text}{fg=gray}
  \setbeamercolor{alerted text}{fg=blue}
  \usebeamercolor{normal text}
  Use filters to manipulate local data \ldots
  \begin{itemize}[<+->]
    \item \alert<1>{create new facts}
  \end{itemize}
  \begin{lstlisting}
  - name: count skills
    set_fact:
      skills_length: "{{ martin.skills | length }}"
  \end{lstlisting}
\end{frame}

\begin{frame}[fragile]
  \frametitle{Typical Filter Use Cases}
  \setbeamercolor{normal text}{fg=gray}
  \setbeamercolor{alerted text}{fg=blue}
  \usebeamercolor{normal text}
  Use filters to manipulate local data \ldots
  \begin{itemize}
    \item {create new facts}
    \item \alert<1>{subset or filter lists}
  \end{itemize}
  \begin{lstlisting}
  - debug:
      var: item
    with_items: "{{ martin.skills }}"
    when: item | match('p.*')

  - debug:
      var: item
    with_items: "{{ martin.skills }}"
    when: not item | match('perl') # same as item != 'perl'
  \end{lstlisting}
\end{frame}

\begin{frame}[fragile]
  \frametitle{Typical Filter Use Cases}
  \setbeamercolor{normal text}{fg=gray}
  \setbeamercolor{alerted text}{fg=blue}
  \usebeamercolor{normal text}
  Use filters to manipulate local data \ldots
  \begin{itemize}
    \item {create new facts}
    \item {subset or filter lists}
    \item \alert<1>{manipulate strings} 
  \end{itemize}
  \begin{lstlisting}
  - debug:
      msg: "{{ martin.job | lower }}"

  - debug:
      var: item
    with_items: "{{ martin.name.split(' ') }}"

  - name: capitalize skills
    debug:
      msg: "{{ item | capitalize }}"
    with_items: "{{ martin.skills }}"
  \end{lstlisting}
\end{frame}

\section{Debugging}

\begin{frame}
  \frametitle{When things go wrong \ldots}
  \begin{itemize}
    \item{debugging \ldots}
  \end{itemize}
\end{frame}

\section{Roll Your Own}

\begin{frame}
  \frametitle{Writing your own filter \ldots}
  \begin{itemize}
    \item{\ldots}
  \end{itemize}
\end{frame}

\section{Summary}

\begin{frame}
  \frametitle{What we covered \ldots}
    \pause{}
  \begin{itemize}
    \item{what filters are}
      \pause{}
    \item{typical places you will use them}
      \pause{}
    \item{getting help when things go wrong}
      \pause{}
    \item{writing your own filter}
  \end{itemize}
\end{frame}

\begin{frame}
  \frametitle{References}
  \begin{itemize}
    \item{ \href{https://www.ansible.com/}{www.ansible.com} }
    \item{ \href{http://jinja.pocoo.org/}{jinja.pocoo.org} }
  \end{itemize}
\end{frame}

\end{document}
