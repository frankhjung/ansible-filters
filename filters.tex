\documentclass[pdf]{beamer}

\mode<presentation>{
  \usetheme{Singapore}          % others: default Singapore Warsaw
  \usecolortheme{dolphin}       % beetle beaver orchid whale dolphin
  \setbeamertemplate{sections/subsections in toc}[ball unnumbered]  % others: circle ball square
	\usepackage{graphicx}
	\graphicspath{ {figures/} }
}

% preamble
\title{How do Jinja2 filters work?}
\subtitle{a short introduction to ansible filters}
\author{Frank Jung}
\institute{frankhjung@linux.com}
\date{\today}
\logo{\includegraphics[height=1.5cm]{logos.png}}

\begin{document}

\begin{frame}
  \titlepage{}
\end{frame}

\begin{frame}{Contents}
  \tableofcontents{}
\end{frame}

\begin{frame}{Topics}
  Topics being covered are \ldots
  \pause{}
  \begin{itemize}
    \item{} what are ansible filters?
      \pause{}
    \item{} what are pipes?
      \pause{}
    \item{} when documentation is not enough
      \pause{}
    \item{} when all else fails
      \pause{}
    \item{} should I roll my own?
  \end{itemize}
\end{frame}

\section{What are Ansible Filters?}
\begin{frame}{What are Ansible Filters?}
  Filters are used to transform data:
  \pause{}
  \begin{enumerate}
    \item{} they can be used inside a template expression
    \pause{}
    \item{} they can be used to manipulate local data
  \end{enumerate}
\end{frame}

\begin{frame}{What are Ansible Filters?}
  Filters are from a fast, flexible, Python template engine called ...
  \pause{}
  \begin{center}
    \textbf{Jinja2} \\
    \begin{figure}
      \includegraphics{jinja-logo.png}
    \end{figure}
    \href{http://jinja.pocoo.org}{jinja.pocoo.org}
  \end{center}
\end{frame}

\begin{frame}{What are Ansible Filters?}
  Important features:
  \begin{enumerate}
    \item{} sand boxed \\
    \pause{} - so can be used to evaluate untrusted code
    \pause{}
    \item{} they are executed on the Ansible controller \\
    \pause{} - \textbf{not} on the task's target host
		\item{} Ansible ships with its own filters \\
    \pause{} - or use standard filters from Jinja2 \\
    \pause{} - or write your own!
  \end{enumerate}
\end{frame}

\section{What are pipes?}
\begin{frame}{Numbered Points}
  Focusing on how filters can be used to manipulate local data ...
  \begin{enumerate}
    \item{} $1^{st}$ item
    \item{} $2^{nd}$ item
    \item{} $3^{rd}$ item
  \end{enumerate}
\end{frame}

% bullet points
\section{Bullet Points}
\begin{frame}{Bullet Points}
  Bullet points
  \begin{itemize}
    \item{} $1^{st}$ item
    \item{} $2^{nd}$ item
    \item{} $3^{rd}$ item
  \end{itemize}
\end{frame}

\end{document}
